\label{chap:PandC}
\newthought{In what relation} do the Communists stand to the proletarians as a
whole?

The Communists do not form a separate party opposed to other
working-class parties.

They have no interests separate and apart from those of the proletariat
as a whole.

They do not set up any sectarian principles of their own, by which to
shape and mould the proletarian movement.

The Communists are distinguished from the other working-class parties
by this only: (1) In the national struggles of the proletarians of the
different countries, they point out and bring to the front the common
interests of the entire proletariat, independently of all nationality.
(2) In the various stages of development which the struggle of the
working class against the bourgeoisie has to pass through, they always
and everywhere represent the interests of the movement as a whole.

The Communists, therefore, are on the one hand, practically, the most
advanced and resolute section of the working-class parties of every
country, that section which pushes forward all others; on the other
hand, theoretically, they have over the great mass of the proletariat
the advantage of clearly understanding the line of march, the
conditions, and the ultimate general results of the proletarian
movement.

The immediate aim of the Communist is the same as that of all the other
proletarian parties: formation of the proletariat into a class,
overthrow of the bourgeois supremacy, conquest of political power by
the proletariat.

The theoretical conclusions of the Communists are in no way based on
ideas or principles that have been invented, or discovered, by this or
that would-be universal reformer. They merely express, in general
terms, actual relations springing from an existing class struggle, from
a historical movement going on under our very eyes. The abolition of
existing property relations is not at all a distinctive feature of
Communism.

All property relations in the past have continually been subject to
historical change consequent upon the change in historical conditions.

The French Revolution, for example, abolished feudal property in favour
of bourgeois property.

The distinguishing feature of Communism is not the abolition of
property generally, but the abolition of bourgeois property. But modern
bourgeois private property is the final and most complete expression of
the system of producing and appropriating products, that is based on
class antagonisms, on the exploitation of the many by the few.

In this sense, the theory of the Communists may be summed up in the
single sentence: Abolition of private property.

We Communists have been reproached with the desire of abolishing the
right of personally acquiring property as the fruit of a man’s own
labour, which property is alleged to be the groundwork of all personal
freedom, activity and independence.

Hard-won, self-acquired, self-earned property! Do you mean the property
of the petty artisan and of the small peasant, a form of property that
preceded the bourgeois form? There is no need to abolish that; the
development of industry has to a great extent already destroyed it, and
is still destroying it daily.

Or do you mean modern bourgeois private property?

But does wage-labour create any property for the labourer? Not a bit.
It creates capital, i.e., that kind of property which exploits
wage-labour, and which cannot increase except upon condition of
begetting a new supply of wage-labour for fresh exploitation. Property,
in its present form, is based on the antagonism of capital and
wage-labour. Let us examine both sides of this antagonism.

To be a capitalist, is to have not only a purely personal, but a social
\textit{status} in production. Capital is a collective product, and only by
the united action of many members, nay, in the last resort, only by the
united action of all members of society, can it be set in motion.

Capital is, therefore, not a personal, it is a social power.

When, therefore, capital is converted into common property, into the
property of all members of society, personal property is not thereby
transformed into social property. It is only the social character of
the property that is changed. It loses its class-character.

Let us now take wage-labour.

The average price of wage-labour is the minimum wage, \textit{i.e.}, that
quantum of the means of subsistence, which is absolutely requisite in
bare existence as a labourer. What, therefore, the wage-labourer
appropriates by means of his labour, merely suffices to prolong and
reproduce a bare existence. We by no means intend to abolish this
personal appropriation of the products of labour, an appropriation that
is made for the maintenance and reproduction of human life, and that
leaves no surplus wherewith to command the labour of others. All that
we want to do away with, is the miserable character of this
appropriation, under which the labourer lives merely to increase
capital, and is allowed to live only in so far as the interest of the
ruling class requires it.

In bourgeois society, living labour is but a means to increase
accumulated labour. In Communist society, accumulated labour is but a
means to widen, to enrich, to promote the existence of the labourer.

In bourgeois society, therefore, the past dominates the present; in
Communist society, the present dominates the past. In bourgeois society
capital is independent and has individuality, while the living person
is dependent and has no individuality.

And the abolition of this state of things is called by the bourgeois,
abolition of individuality and freedom! And rightly so. The abolition
of bourgeois individuality, bourgeois independence, and bourgeois
freedom is undoubtedly aimed at.

By freedom is meant, under the present bourgeois conditions of
production, free trade, free selling and buying.

But if selling and buying disappears, free selling and buying
disappears also. This talk about free selling and buying, and all the
other \enquote{brave words} of our bourgeoisie about freedom in general, have a
meaning, if any, only in contrast with restricted selling and buying,
with the fettered traders of the Middle Ages, but have no meaning when
opposed to the Communistic abolition of buying and selling, of the
bourgeois conditions of production, and of the bourgeoisie itself.

You are horrified at our intending to do away with private property.
But in your existing society, private property is already done away
with for nine-tenths of the population; its existence for the few is
solely due to its non-existence in the hands of those nine-tenths. You
reproach us, therefore, with intending to do away with a form of
property, the necessary condition for whose existence is the
non-existence of any property for the immense majority of society.

In one word, you reproach us with intending to do away with your
property. Precisely so; that is just what we intend.

From the moment when labour can no longer be converted into capital,
money, or rent, into a social power capable of being monopolised,
\textit{i.e.}, from the moment when individual property can no longer be
transformed into bourgeois property, into capital, from that moment,
you say individuality vanishes.

You must, therefore, confess that by \enquote{individual} you mean no other
person than the bourgeois, than the middle-class owner of property.
This person must, indeed, be swept out of the way, and made impossible.

Communism deprives no man of the power to appropriate the products of
society; all that it does is to deprive him of the power to subjugate
the labour of others by means of such appropriation.

It has been objected that upon the abolition of private property all
work will cease, and universal laziness will overtake us.

According to this, bourgeois society ought long ago to have gone to the
dogs through sheer idleness; for those of its members who work, acquire
nothing, and those who acquire anything, do not work. The whole of this
objection is but another expression of the tautology: that there can no
longer be any wage-labour when there is no longer any capital.

All objections urged against the Communistic mode of producing and
appropriating material products, have, in the same way, been urged
against the Communistic modes of producing and appropriating
intellectual products. Just as, to the bourgeois, the disappearance of
class property is the disappearance of production itself, so the
disappearance of class culture is to him identical with the
disappearance of all culture.

That culture, the loss of which he laments, is, for the enormous
majority, a mere training to act as a machine.

But don’t wrangle with us so long as you apply, to our intended
abolition of bourgeois property, the standard of your bourgeois notions
of freedom, culture, law, etc. Your very ideas are but the outgrowth of
the conditions of your bourgeois production and bourgeois property,
just as your jurisprudence is but the will of your class made into a
law for all, a will, whose essential character and direction are
determined by the economical conditions of existence of your class.

The selfish misconception that induces you to transform into eternal
laws of nature and of reason, the social forms springing from your
present mode of production and form of property—historical relations
that rise and disappear in the progress of production—this
misconception you share with every ruling class that has preceded you.
What you see clearly in the case of ancient property, what you admit in
the case of feudal property, you are of course forbidden to admit in
the case of your own bourgeois form of property.

Abolition of the family! Even the most radical flare up at this
infamous proposal of the Communists.

On what foundation is the present family, the bourgeois family, based?
On capital, on private gain. In its completely developed form this
family exists only among the bourgeoisie. But this state of things
finds its complement in the practical absence of the family among the
proletarians, and in public prostitution.

The bourgeois family will vanish as a matter of course when its
complement vanishes, and both will vanish with the vanishing of
capital.

Do you charge us with wanting to stop the exploitation of children by
their parents? To this crime we plead guilty.

But, you will say, we destroy the most hallowed of relations, when we
replace home education by social.

And your education! Is not that also social, and determined by the
social conditions under which you educate, by the intervention, direct
or indirect, of society, by means of schools, etc.? The Communists have
not invented the intervention of society in education; they do but seek
to alter the character of that intervention, and to rescue education
from the influence of the ruling class.

The bourgeois clap-trap about the family and education, about the
hallowed co-relation of parent and child, becomes all the more
disgusting, the more, by the action of Modern Industry, all family ties
among the proletarians are torn asunder, and their children transformed
into simple articles of commerce and instruments of labour.

But you Communists would introduce community of women, screams the
whole bourgeoisie in chorus.

The bourgeois sees in his wife a mere instrument of production. He
hears that the instruments of production are to be exploited in common,
and, naturally, can come to no other conclusion than that the lot of
being common to all will likewise fall to the women.

He has not even a suspicion that the real point is to do away with the
status of women as mere instruments of production.

For the rest, nothing is more ridiculous than the virtuous indignation
of our bourgeois at the community of women which, they pretend, is to
be openly and officially established by the Communists. The Communists
have no need to introduce community of women; it has existed almost
from time immemorial.

Our bourgeois, not content with having the wives and daughters of their
proletarians at their disposal, not to speak of common prostitutes,
take the greatest pleasure in seducing each other’s wives.

Bourgeois marriage is in reality a system of wives in common and thus,
at the most, what the Communists might possibly be reproached with, is
that they desire to introduce, in substitution for a hypocritically
concealed, an openly legalised community of women. For the rest, it is
self-evident that the abolition of the present system of production
must bring with it the abolition of the community of women springing
from that system, \textit{i.e.}, of prostitution both public and private.

The Communists are further reproached with desiring to abolish
countries and nationality.

The working men have no country. We cannot take from them what they
have not got. Since the proletariat must first of all acquire political
supremacy, must rise to be the leading class of the nation, must
constitute itself \textit{the} nation, it is, so far, itself national, though
not in the bourgeois sense of the word.

National differences and antagonisms between peoples are daily more and
more vanishing, owing to the development of the bourgeoisie, to freedom
of commerce, to the world-market, to uniformity in the mode of
production and in the conditions of life corresponding thereto.

The supremacy of the proletariat will cause them to vanish still
faster. United action, of the leading civilised countries at least, is
one of the first conditions for the emancipation of the proletariat.

In proportion as the exploitation of one individual by another is put
an end to, the exploitation of one nation by another will also be put
an end to. In proportion as the antagonism between classes within the
nation vanishes, the hostility of one nation to another will come to an
end.

The charges against Communism made from a religious, a philosophical,
and, generally, from an ideological standpoint, are not deserving of
serious examination.

Does it require deep intuition to comprehend that man’s ideas, views
and conceptions, in one word, man’s consciousness, changes with every
change in the conditions of his material existence, in his social
relations and in his social life?

What else does the history of ideas prove, than that intellectual
production changes its character in proportion as material production
is changed? The ruling ideas of each age have ever been the ideas of
its ruling class.

When people speak of ideas that revolutionise society, they do but
express the fact, that within the old society, the elements of a new
one have been created, and that the dissolution of the old ideas keeps
even pace with the dissolution of the old conditions of existence.

When the ancient world was in its last throes, the ancient religions
were overcome by Christianity. When Christian ideas succumbed in the
18th century to rationalist ideas, feudal society fought its death
battle with the then revolutionary bourgeoisie. The ideas of religious
liberty and freedom of conscience merely gave expression to the sway of
free competition within the domain of knowledge.

\enquote{Undoubtedly,} it will be said, \enquote{religious, moral, philosophical and
juridical ideas have been modified in the course of historical
development. But religion, morality philosophy, political science, and
law, constantly survived this change.}

\enquote{There are, besides, eternal truths, such as Freedom, Justice, etc.
that are common to all states of society. But Communism abolishes
eternal truths, it abolishes all religion, and all morality, instead of
constituting them on a new basis; it therefore acts in contradiction to
all past historical experience.}

What does this accusation reduce itself to? The history of all past
society has consisted in the development of class antagonisms,
antagonisms that assumed different forms at different epochs.

But whatever form they may have taken, one fact is common to all past
ages, viz., the exploitation of one part of society by the other. No
wonder, then, that the social consciousness of past ages, despite all
the multiplicity and variety it displays, moves within certain common
forms, or general ideas, which cannot completely vanish except with the
total disappearance of class antagonisms.

The Communist revolution is the most radical rupture with traditional
property relations; no wonder that its development involves the most
radical rupture with traditional ideas.

But let us have done with the bourgeois objections to Communism.

We have seen above, that the first step in the revolution by the
working class, is to raise the proletariat to the position of ruling as
to win the battle of democracy.

The proletariat will use its political supremacy to wrest, by degrees,
all capital from the bourgeoisie, to centralise all instruments of
production in the hands of the State, \textit{i.e.}, of the proletariat
organised as the ruling class; and to increase the total of productive
forces as rapidly as possible.

Of course, in the beginning, this cannot be effected except by means of
despotic inroads on the rights of property, and on the conditions of
bourgeois production; by means of measures, therefore, which appear
economically insufficient and untenable, but which, in the course of
the movement, outstrip themselves, necessitate further inroads upon the
old social order, and are unavoidable as a means of entirely
revolutionising the mode of production.

These measures will of course be different in different countries.

Nevertheless in the most advanced countries, the following will be
pretty generally applicable.

\begin{enumerate}
\item Abolition of property in land and application of all rents of land
to public purposes.

\item A heavy progressive or graduated income tax.

\item Abolition of all right of inheritance.

\item Confiscation of the property of all emigrants and rebels.

\item Centralisation of credit in the hands of the State, by means of a
national bank with State capital and an exclusive monopoly.

\item Centralisation of the means of communication and transport in the
hands of the State.

\item Extension of factories and instruments of production owned by the
State; the bringing into cultivation of waste-lands, and the
improvement of the soil generally in accordance with a common plan.

\item Equal liability of all to labour. Establishment of industrial
armies, especially for agriculture.

\item Combination of agriculture with manufacturing industries; gradual
abolition of the distinction between town and country, by a more
equable distribution of the population over the country.

\item Free education for all children in public schools.     Abolition of
children’s factory labour in its present form.     Combination of
education with industrial production, \&c., \&c.
\end{enumerate}

When, in the course of development, class distinctions have
disappeared, and all production has been concentrated in the hands of a
vast association of the whole nation, the public power will lose its
political character. Political power, properly so called, is merely the
organised power of one class for oppressing another. If the proletariat
during its contest with the bourgeoisie is compelled, by the force of
circumstances, to organise itself as a class, if, by means of a
revolution, it makes itself the ruling class, and, as such, sweeps away
by force the old conditions of production, then it will, along with
these conditions, have swept away the conditions for the existence of
class antagonisms and of classes generally, and will thereby have
abolished its own supremacy as a class.

In place of the old bourgeois society, with its classes and class
antagonisms, we shall have an association, in which the free
development of each is the condition for the free development of all.


