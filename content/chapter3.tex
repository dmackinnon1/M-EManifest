\section*{Reactionary Socialism}

\subsection*{Feudal Socialism}

\newthought{Owing to their historical position}, it became the vocation of the
aristocracies of France and England to write pamphlets against modern
bourgeois society. In the French revolution of July 1830, and in the
English reform agitation, these aristocracies again succumbed to the
hateful upstart. Thenceforth, a serious political contest was
altogether out of the question. A literary battle alone remained
possible. But even in the domain of literature the old cries of the
restoration period had become impossible.

In order to arouse sympathy, the aristocracy were obliged to lose
sight, apparently, of their own interests, and to formulate their
indictment against the bourgeoisie in the interest of the exploited
working class alone. Thus the aristocracy took their revenge by singing
lampoons on their new master, and whispering in his ears sinister
prophecies of coming catastrophe.

In this way arose Feudal Socialism: half lamentation, half lampoon;
half echo of the past, half menace of the future; at times, by its
bitter, witty and incisive criticism, striking the bourgeoisie to the
very heart’s core; but always ludicrous in its effect, through total
incapacity to comprehend the march of modern history.

The aristocracy, in order to rally the people to them, waved the
proletarian alms-bag in front for a banner. But the people, so often as
it joined them, saw on their hindquarters the old feudal coats of arms,
and deserted with loud and irreverent laughter.

One section of the French Legitimists and \enquote{Young England} exhibited
this spectacle.

In pointing out that their mode of exploitation was different to that
of the bourgeoisie, the feudalists forget that they exploited under
circumstances and conditions that were quite different, and that are
now antiquated. In showing that, under their rule, the modern
proletariat never existed, they forget that the modern bourgeoisie is
the necessary offspring of their own form of society.

For the rest, so little do they conceal the reactionary character of
their criticism that their chief accusation against the bourgeoisie
amounts to this, that under the bourgeois \textit{regime} a class is being
developed, which is destined to cut up root and branch the old order of
society.

What they upbraid the bourgeoisie with is not so much that it creates a
proletariat, as that it creates a \textit{revolutionary} proletariat.

In political practice, therefore, they join in all coercive measures
against the working class; and in ordinary life, despite their high
falutin phrases, they stoop to pick up the golden apples dropped from
the tree of industry, and to barter truth, love, and honour for traffic
in wool, beetroot-sugar, and potato spirits.

As the parson has ever gone hand in hand with the landlord, so has
Clerical Socialism with Feudal Socialism.

Nothing is easier than to give Christian asceticism a Socialist tinge.
Has not Christianity declaimed against private property, against
marriage, against the State? Has it not preached in the place of these,
charity and poverty, celibacy and mortification of the flesh, monastic
life and Mother Church? Christian Socialism is but the holy water with
which the priest consecrates the heart-burnings of the aristocrat.

\subsection*{Petty-Bourgeois Socialism}
\newthought{The feudal aristocracy} was not the only class that was ruined by the
bourgeoisie, not the only class whose conditions of existence pined and
perished in the atmosphere of modern bourgeois society. The mediaeval
burgesses and the small peasant proprietors were the precursors of the
modern bourgeoisie. In those countries which are but little developed,
industrially and commercially, these two classes still vegetate side by
side with the rising bourgeoisie.

In countries where modern civilisation has become fully developed, a
new class of petty bourgeois has been formed, fluctuating between
proletariat and bourgeoisie and ever renewing itself as a supplementary
part of bourgeois society. The individual members of this class,
however, are being constantly hurled down into the proletariat by the
action of competition, and, as modern industry develops, they even see
the moment approaching when they will completely disappear as an
independent section of modern society, to be replaced, in manufactures,
agriculture and commerce, by overlookers, bailiffs and shopmen.

In countries like France, where the peasants constitute far more than
half of the population, it was natural that writers who sided with the
proletariat against the bourgeoisie, should use, in their criticism of
the bourgeois \textit{regime}, the standard of the peasant and petty
bourgeois, and from the standpoint of these intermediate classes should
take up the cudgels for the working class. Thus arose petty-bourgeois
Socialism. Sismondi was the head of this school, not only in France but
also in England.

This school of Socialism dissected with great acuteness the
contradictions in the conditions of modern production. It laid bare the
hypocritical apologies of economists. It proved, incontrovertibly, the
disastrous effects of machinery and division of labour; the
concentration of capital and land in a few hands; overproduction and
crises; it pointed out the inevitable ruin of the petty bourgeois and
peasant, the misery of the proletariat, the anarchy in production, the
crying inequalities in the distribution of wealth, the industrial war
of extermination between nations, the dissolution of old moral bonds,
of the old family relations, of the old nationalities.

In its positive aims, however, this form of Socialism aspires either to
restoring the old means of production and of exchange, and with them
the old property relations, and the old society, or to cramping the
modern means of production and of exchange, within the framework of the
old property relations that have been, and were bound to be, exploded
by those means. In either case, it is both reactionary and Utopian.

Its last words are: corporate guilds for manufacture, patriarchal
relations in agriculture.

Ultimately, when stubborn historical facts had dispersed all
intoxicating effects of self-deception, this form of Socialism ended in
a miserable fit of the blues.

\subsection*{German, or ``True,'' Socialism}
\newthought{The Socialist and Communist literature} of France, a literature that
originated under the pressure of a bourgeoisie in power, and that was
the expression of the struggle against this power, was introduced into
Germany at a time when the bourgeoisie, in that country, had just begun
its contest with feudal absolutism.

German philosophers, would-be philosophers, and \textit{beaux esprits},
eagerly seized on this literature, only forgetting, that when these
writings immigrated from France into Germany, French social conditions
had not immigrated along with them. In contact with German social
conditions, this French literature lost all its immediate practical
significance, and assumed a purely literary aspect. Thus, to the German
philosophers of the eighteenth century, the demands of the first French
Revolution were nothing more than the demands of \enquote{Practical Reason} in
general, and the utterance of the will of the revolutionary French
bourgeoisie signified in their eyes the law of pure Will, of Will as it
was bound to be, of true human Will generally.

The world of the German \textit{literati} consisted solely in bringing the new
French ideas into harmony with their ancient philosophical conscience,
or rather, in annexing the French ideas without deserting their own
philosophic point of view.

This annexation took place in the same way in which a foreign language
is appropriated, namely, by translation.

It is well known how the monks wrote silly lives of Catholic Saints
\textit{over} the manuscripts on which the classical works of ancient
heathendom had been written. The German \textit{literati} reversed this
process with the profane French literature. They wrote their
philosophical nonsense beneath the French original. For instance,
beneath the French criticism of the economic functions of money, they
wrote \enquote{Alienation of Humanity,} and beneath the French criticism of the
bourgeois State they wrote \enquote{dethronement of the Category of the
General,} and so forth.

The introduction of these philosophical phrases at the back of the
French historical criticisms they dubbed \enquote{Philosophy of Action,} \enquote{True
Socialism,} \enquote{German Science of Socialism,} \enquote{Philosophical Foundation of
Socialism,} and so on.

The French Socialist and Communist literature was thus completely
emasculated. And, since it ceased in the hands of the German to express
the struggle of one class with the other, he felt conscious of having
overcome \enquote{French one-sidedness} and of representing, not true
requirements, but the requirements of truth; not the interests of the
proletariat, but the interests of Human Nature, of Man in general, who
belongs to no class, has no reality, who exists only in the misty realm
of philosophical fantasy.

This German Socialism, which took its schoolboy task so seriously and
solemnly, and extolled its poor stock-in-trade in such mountebank
fashion, meanwhile gradually lost its pedantic innocence.

The fight of the German, and especially, of the Prussian bourgeoisie,
against feudal aristocracy and absolute monarchy, in other words, the
liberal movement, became more earnest.

By this, the long wished-for opportunity was offered to \enquote{True}
Socialism of confronting the political movement with the Socialist
demands, of hurling the traditional anathemas against liberalism,
against representative government, against bourgeois competition,
bourgeois freedom of the press, bourgeois legislation, bourgeois
liberty and equality, and of preaching to the masses that they had
nothing to gain, and everything to lose, by this bourgeois movement.
German Socialism forgot, in the nick of time, that the French
criticism, whose silly echo it was, presupposed the existence of modern
bourgeois society, with its corresponding economic conditions of
existence, and the political constitution adapted thereto, the very
things whose attainment was the object of the pending struggle in
Germany.

To the absolute governments, with their following of parsons,
professors, country squires and officials, it served as a welcome
scarecrow against the threatening bourgeoisie.

It was a sweet finish after the bitter pills of floggings and bullets
with which these same governments, just at that time, dosed the German
working-class risings.

While this \enquote{True} Socialism thus served the governments as a weapon for
fighting the German bourgeoisie, it, at the same time, directly
represented a reactionary interest, the interest of the German
Philistines. In Germany the \textit{petty bourgeois} class, a \textit{relique} of the
sixteenth century, and since then constantly cropping up again under
various forms, is the real social basis of the existing state of
things.

To preserve this class is to preserve the existing state of things in
Germany. The industrial and political supremacy of the bourgeoisie
threatens it with certain destruction; on the one hand, from the
concentration of capital; on the other, from the rise of a
revolutionary proletariat. \enquote{True} Socialism appeared to kill these two
birds with one stone. It spread like an epidemic.

The robe of speculative cobwebs, embroidered with flowers of rhetoric,
steeped in the dew of sickly sentiment, this transcendental robe in
which the German Socialists wrapped their sorry \enquote{eternal truths,} all
skin and bone, served to wonderfully increase the sale of their goods
amongst such a public. And on its part, German Socialism recognised,
more and more, its own calling as the bombastic representative of the
petty-bourgeois Philistine.

It proclaimed the German nation to be the model nation, and the German
petty Philistine to be the typical man. To every villainous meanness of
this model man it gave a hidden, higher, Socialistic interpretation,
the exact contrary of its real character. It went to the extreme length
of directly opposing the \enquote{brutally destructive} tendency of Communism,
and of proclaiming its supreme and impartial contempt of all class
struggles. With very few exceptions, all the so-called Socialist and
Communist publications that now (1847) circulate in Germany belong to
the domain of this foul and enervating literature.

\section*{Conservative or Bourgeois Socialism}
\newthought{A part of the bourgeoisie} is desirous of redressing social grievances,
in order to secure the continued existence of bourgeois society.

To this section belong economists, philanthropists, humanitarians,
improvers of the condition of the working class, organisers of charity,
members of societies for the prevention of cruelty to animals,
temperance fanatics, hole-and-corner reformers of every imaginable
kind. This form of Socialism has, moreover, been worked out into
complete systems.

We may cite Proudhon’s \textit{Philosophie de la Misère} as an example of this
form.

The Socialistic bourgeois want all the advantages of modern social
conditions without the struggles and dangers necessarily resulting
therefrom. They desire the existing state of society minus its
revolutionary and disintegrating elements. They wish for a bourgeoisie
without a proletariat. The bourgeoisie naturally conceives the world in
which it is supreme to be the best; and bourgeois Socialism develops
this comfortable conception into various more or less complete systems.
In requiring the proletariat to carry out such a system, and thereby to
march straightway into the social New Jerusalem, it but requires in
reality, that the proletariat should remain within the bounds of
existing society, but should cast away all its hateful ideas concerning
the bourgeoisie.

A second and more practical, but less systematic, form of this
Socialism sought to depreciate every revolutionary movement in the eyes
of the working class, by showing that no mere political reform, but
only a change in the material conditions of existence, in economic
relations, could be of any advantage to them. By changes in the
material conditions of existence, this form of Socialism, however, by
no means understands abolition of the bourgeois relations of
production, an abolition that can be effected only by a revolution, but
administrative reforms, based on the continued existence of these
relations; reforms, therefore, that in no respect affect the relations
between capital and labour, but, at the best, lessen the cost, and
simplify the administrative work, of bourgeois government.

Bourgeois Socialism attains adequate expression, when, and only when,
it becomes a mere figure of speech.

Free trade: for the benefit of the working class. Protective duties:
for the benefit of the working class. Prison Reform: for the benefit of
the working class. This is the last word and the only seriously meant
word of bourgeois Socialism.

It is summed up in the phrase: the bourgeois is a bourgeois—for the
benefit of the working class.

\section*{Critical-Utopian Socialism and Communism}

\newthought{We do not} here refer to that literature which, in every great modern
revolution, has always given voice to the demands of the proletariat,
such as the writings of Babeuf and others.

The first direct attempts of the proletariat to attain its own ends,
made in times of universal excitement, when feudal society was being
overthrown, these attempts necessarily failed, owing to the then
undeveloped state of the proletariat, as well as to the absence of the
economic conditions for its emancipation, conditions that had yet to be
produced, and could be produced by the impending bourgeois epoch alone.
The revolutionary literature that accompanied these first movements of
the proletariat had necessarily a reactionary character. It inculcated
universal asceticism and social levelling in its crudest form.

The Socialist and Communist systems properly so called, those of
Saint-Simon, Fourier, Owen and others, spring into existence in the
early undeveloped period, described above, of the struggle between
proletariat and bourgeoisie (see Section 1. Bourgeois and
Proletarians).

The founders of these systems see, indeed, the class antagonisms, as
well as the action of the decomposing elements, in the prevailing form
of society. But the proletariat, as yet in its infancy, offers to them
the spectacle of a class without any historical initiative or any
independent political movement.

Since the development of class antagonism keeps even pace with the
development of industry, the economic situation, as they find it, does
not as yet offer to them the material conditions for the emancipation
of the proletariat. They therefore search after a new social science,
after new social laws, that are to create these conditions.

Historical action is to yield to their personal inventive action,
historically created conditions of emancipation to fantastic ones, and
the gradual, spontaneous class-organisation of the proletariat to the
organisation of society specially contrived by these inventors. Future
history resolves itself, in their eyes, into the propaganda and the
practical carrying out of their social plans.

In the formation of their plans they are conscious of caring chiefly
for the interests of the working class, as being the most suffering
class. Only from the point of view of being the most suffering class
does the proletariat exist for them.

The undeveloped state of the class struggle, as well as their own
surroundings, causes Socialists of this kind to consider themselves far
superior to all class antagonisms. They want to improve the condition
of every member of society, even that of the most favoured. Hence, they
habitually appeal to society at large, without distinction of class;
nay, by preference, to the ruling class. For how can people, when once
they understand their system, fail to see in it the best possible plan
of the best possible state of society?

Hence, they reject all political, and especially all revolutionary,
action; they wish to attain their ends by peaceful means, and
endeavour, by small experiments, necessarily doomed to failure, and by
the force of example, to pave the way for the new social Gospel.

Such fantastic pictures of future society, painted at a time when the
proletariat is still in a very undeveloped state and has but a
fantastic conception of its own position correspond with the first
instinctive yearnings of that class for a general reconstruction of
society.

But these Socialist and Communist publications contain also a critical
element. They attack every principle of existing society. Hence they
are full of the most valuable materials for the enlightenment of the
working class. The practical measures proposed in them—such as the
abolition of the distinction between town and country, of the family,
of the carrying on of industries for the account of private
individuals, and of the wage system, the proclamation of social
harmony, the conversion of the functions of the State into a mere
superintendence of production, all these proposals, point solely to the
disappearance of class antagonisms which were, at that time, only just
cropping up, and which, in these publications, are recognised in their
earliest, indistinct and undefined forms only. These proposals,
therefore, are of a purely Utopian character.

The significance of Critical-Utopian Socialism and Communism bears an
inverse relation to historical development. In proportion as the modern
class struggle develops and takes definite shape, this fantastic
standing apart from the contest, these fantastic attacks on it, lose
all practical value and all theoretical justification. Therefore,
although the originators of these systems were, in many respects,
revolutionary, their disciples have, in every case, formed mere
reactionary sects. They hold fast by the original views of their
masters, in opposition to the progressive historical development of the
proletariat. They, therefore, endeavour, and that consistently, to
deaden the class struggle and to reconcile the class antagonisms. They
still dream of experimental realisation of their social Utopias, of
founding isolated \enquote{phalansteres,} of establishing \enquote{Home Colonies,} of
setting up a \enquote{Little Icaria}—duodecimo editions of the New
Jerusalem—and to realise all these castles in the air, they are
compelled to appeal to the feelings and purses of the bourgeois. By
degrees they sink into the category of the reactionary conservative
Socialists depicted above, differing from these only by more systematic
pedantry, and by their fanatical and superstitious belief in the
miraculous effects of their social science.

They, therefore, violently oppose all political action on the part of
the working class; such action, according to them, can only result from
blind unbelief in the new Gospel.

The Owenites in England, and the Fourierists in France, respectively,
oppose the Chartists and the \enquote{Réformistes.}



