\newthought{Section \ref{chap:PandC} has made clear} the relations of the Communists to the
existing working-class parties, such as the Chartists in England and
the Agrarian Reformers in America.

The Communists fight for the attainment of the immediate aims, for the
enforcement of the momentary interests of the working class; but in the
movement of the present, they also represent and take care of the
future of that movement. In France the Communists ally themselves with
the Social-Democrats, against the conservative and radical bourgeoisie,
reserving, however, the right to take up a critical position in regard
to phrases and illusions traditionally handed down from the great
Revolution.

In Switzerland they support the Radicals, without losing sight of the
fact that this party consists of antagonistic elements, partly of
Democratic Socialists, in the French sense, partly of radical
bourgeois.

In Poland they support the party that insists on an agrarian revolution
as the prime condition for national emancipation, that party which
fomented the insurrection of Cracow in 1846.

In Germany they fight with the bourgeoisie whenever it acts in a
revolutionary way, against the absolute monarchy, the feudal
squirearchy, and the petty bourgeoisie.

But they never cease, for a single instant, to instil into the working
class the clearest possible recognition of the hostile antagonism
between bourgeoisie and proletariat, in order that the German workers
may straightaway use, as so many weapons against the bourgeoisie, the
social and political conditions that the bourgeoisie must necessarily
introduce along with its supremacy, and in order that, after the fall
of the reactionary classes in Germany, the fight against the
bourgeoisie itself may immediately begin.

The Communists turn their attention chiefly to Germany, because that
country is on the eve of a bourgeois revolution that is bound to be
carried out under more advanced conditions of European civilisation,
and with a much more developed proletariat, than that of England was in
the seventeenth, and of France in the eighteenth century, and because
the bourgeois revolution in Germany will be but the prelude to an
immediately following proletarian revolution.

In short, the Communists everywhere support every revolutionary
movement against the existing social and political order of things.

In all these movements they bring to the front, as the leading question
in each, the property question, no matter what its degree of
development at the time.

Finally, they labour everywhere for the union and agreement of the
democratic parties of all countries.

The Communists disdain to conceal their views and aims. They openly
declare that their ends can be attained only by the forcible overthrow
of all existing social conditions. Let the ruling classes tremble at a
Communistic revolution. The proletarians have nothing to lose but their
chains. They have a world to win.

{

\centering
WORKING MEN OF ALL COUNTRIES,\\ UNITE!

}
